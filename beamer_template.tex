\documentclass{beamer}
%
% Choose how your presentation looks.
%
% For more themes, color themes and font themes, see:
% http://deic.uab.es/~iblanes/beamer_gallery/index_by_theme.html
%
\mode<presentation>
{
  \usetheme{default}      % or try Darmstadt, Madrid, Warsaw, ...
  \usecolortheme{default} % or try albatross, beaver, crane, ...
  \usefonttheme{default}  % or try serif, structurebold, ...
  \setbeamertemplate{navigation symbols}{}
  \setbeamertemplate{caption}[numbered]
} 

\usepackage[english]{babel}
\usepackage[utf8x]{inputenc}

\title[ext talk]{Extrusion of a Vortex Lattice}
\author{Jonathan Watkins}
\institute{University of Birmingham}
\date{2014}

\begin{document}

\begin{frame}
  \titlepage
\end{frame}

% Uncomment these lines for an automatically generated outline.
%\begin{frame}{Outline}
%  \tableofcontents
%\end{frame}

\section{Introduction}

\begin{frame}{Introduction}

\begin{itemize}
  \item Introduction
  \item Physical System
  \item Model
  \item How does the 'solid' flow?
  \item Extension to hollow wire
  \item Summary
\end{itemize}

%\vskip 1cm
%\begin{block}{Examples}
%Some examples of commonly used commands and features are included, to help you get started.
%\end{block}

\end{frame}

\section{Physical System}

\begin{frame}{Physical System}

% Commands to include a figure:
%\begin{figure}
%\includegraphics[width=\textwidth]{your-figure's-file-name}
%\caption{\label{fig:your-figure}Caption goes here.}
%\end{figure}

\end{frame}

\section{Model}

\begin{frame}{Model}


\end{frame}



\section{How does the 'solid' flow}

\begin{frame}{How does the 'solid' flow}


\end{frame}

\section{Extension to hollow wire}

\begin{frame}{Extension to hollow wire}


\end{frame}

\section{Summary}

\begin{frame}{Summary}


\end{frame}

\end{document}
